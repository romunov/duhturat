\section{RAZPRAVA}
Pomembno orodje za raziskovanje nekaterih osnovnih parametrov populacij, kot je njena velikost, so metode lova-ponovnega ulova. Kot velja za vse metode, imajo tudi te nekatere predpostavke, ki nas do neke mere omejujejo. V tem delu smo se osredotočili na kršenje predpostavk zaprtosti populacije in enake ulovljivosti zaradi prihajanja in odhajanja osebkov v in iz območja vzorčenja. To prehajanje imenujemo učinek roba. Posledice učinka roba smo poskusili omiliti s popravkom, ki naj bi po našem mnenju opisoval ulovljivost osebkov v območju vzorčenja.

S pomočjo simulacij in klasičnih modelov za zaprte populacije (Hugginsov model) smo preverili delovanje predlaganega popravka. Vsakemu osebku smo izračunali individualno spremenljivko, ki je odvisna od lege vzorcev (centroida) v prostoru in predpostavljene funkcije gibanja. Gibanje okoli centroida smo opisali s pomočjo dveh porazdelitev. Prva je dvorazsežna normalna porazdelitev, za katero smo uporabili simuliran standardni odklon. Ta predstavlja t. i. ``zlati standard'', saj bi morala najbolje opisati gibanje osebkov. Druga porazdelitev pa je odvisna od premikov osebkov, katerim priležemo rahlo prilagojeno kumulativno porazdelitev Weibullove funkcije. Ta način ne zahteva dodatnih informacij, ki bi jih sicer lahko pridobili z dodatnimi metodami (npr. opremljanje osebkov s telemetrijskimi ovratnicami \citep{ivan-et-al-2013-aux}).

Opisovanje gibanja po (pol)normalni porazdelitvi je v simulacijskih študijah relativno pogosto \citep{bolker_ecological_2008, ivan_using_2013}, saj omogoča razumevanje gibanja in je hkrati (lahko) dovolj kompleksno, da opiše neke splošne lastnosti osebkov, kot se gibljejo v prostoru. Seveda pa je to poenostavitev, za katero ni nujno, da opiše gibanje osebkov dovolj dobro. Za bolj kompleksna gibanja raziskovalci uporabljajo realne podatke, ki jih naključno vzorčijo, in tako simulirajo gibanje osebkov (npr. \citealp{manning_estimating_2010}). Tako generirani podatki nam res dajo bolj realen opis gibanja, vendar pa je možno, da dobimo zaradi nereprezentativnega vzorčenja pristranske rezultate. Tak opis gibanja ni nujno prenosljiv med sezonami ali območji.

V tem delu smo primerjali različne scenarije, kjer smo spreminjali velikost domačega okoliša, število odlovnih intervalov, število simuliranih osebkov (gostoto) in ulovljivost. Velikost in obliko vzorčenega območja smo ohranjali konstantni, ker nas v simulacijah bolj zanima relativen odnos med velikostjo domačega okoliša in velikostjo območja vzorčenja. Rezultate simulacij smo primerjali s pomočjo treh modelov. Dva sta Hugginsova modela, $M_0$, kjer predpostavljamo enako ulovljivost vseh osebkov, in $M_{sp}$, kjer predpostavljamo, da se ulovljivost spreminja z individualno spremenljivko. Tretji model je TIRM, ki predpostavlja, da imajo osebki eno od dveh ulovljivosti.

\subsection{PRISTRANSKOST OCENE ULOVLJIVOSTI}
Ker območje vzorčenja ne zaobjame populacije v celoti, nekateri osebki prehajajo rob vzorčenja. Pri lovu v simulaciji smo v primeru, da smo tak osebek ujeli zunaj območja vzorčenja, ta vzorec krnili. Brez krnjenja bi bila ocenjena ulovljivost nepristranska, blizu simulirane vrednosti zaznavnosti. Ker smo točke, ki se niso nahajale znotraj območja vzorčenja krnili, smo to ulovljivost za robne osebke spremenili - jo zmanjšali. To je razvidno iz rezultatov (npr. slika \ref{sli:slika5}), kjer se odmik ocenjene ulovljivosti odmakne od simulirane, predvsem na račun velikosti domačega okoliša glede na velikost območja vzorčenja. Ko je domači okoliš relativno majhen, so te kršitve relativno majhne, a še vedno prisotne.

Zaradi krnenja ulovljivosti v tej simulaciji nismo popolnoma nadzorovali zaradi česar je težko trditi kater model v povprečju bolje oceni ulovljivost. Vendar pa je razlika med ocenama modelov $M_0$ in $M_{sp}$ tako majhna, da nima vpliva na končni rezultat. Ta podobnost se vidi v pravem testu zanesljivosti, ko smo ocenili gostoto, kjer med modeloma ni praktične razlike.

Ker gre za razmerje med ocenjeno in simulirano vrednostjo, so vrednosti bližje 1 bolj podobne simulirani vrednosti. Odstopanje od simulirane vrednosti je stalno glede na uporabljen model in je manjše za model $M_0$ (bližje pravi simulirani vrednosti) kot za model $M_{sp}$.
Prvi model je preprost, za vse osebke predpostavi enako ulovljivost in ``ne ve'', da so bili podatki krnjeni. Drugi model ``ve'', da smo podatke krnili, in do neke mere to poskuša nadoknaditi s pomočjo individualne spremenljivke. Ta spremenljivka naj bi opisovala pojav, ki spreminja ulovljivost osebka - v našem primeru opisuje, do kakšne mere osebki niso na voljo za vzorčenje zaradi prehajanja roba območja vzorčenja. V tem oziru kaže, da individualna spremenljivka izboljša model. Ocena se izboljša predvsem za simulacije z manj osebki in manj odlovnimi intervali. Razlika med modeloma je komaj opazna, ko pa število osebkov in število odlovnih intervalov povečujemo, razlik med modeloma praktično ni več.

Odstopanje od simulirane vrednosti je različno. Odstopanje za 75 \% ni nenavadno za tiste simulacije, kjer je bila površina domačega okoliša približno tako velika kot območje vzorčenja. S pomanjševanjem površine domačega okoliša relativno na površino območja vzorčenja (razmerje se manjša) pa se ta ocena izboljšuje in v najboljšem primeru odstopa le približno 25 \%. Zaradi krnjenja je manj verjetno, da bi ocenjene vrednosti v povprečju kadarkoli dosegle teoretično vrednost razmerja 1.

Iz razmerja ocenjene in simulirane ulovljivosti lahko opazimo, da se je variabilnost zmanjšala, ko smo povečali število odlovnih intervalov, nekoliko pa tudi zaradi večjega števila simuliranih osebkov. Manjša variabilnost je na račun tega, da smo v model vključili več podatkov. Takrat so ocene bolj natančne in je posledično ocena parametrov manj variabilna. Ta pojav nam je lahko v pomoč pri načrtovanju poskusa, kjer se odločamo, ali napor vložiti v večanje ulovljivosti osebkov, v večanje števila odlovnih intervalov ali obojega.

Kot osnovo, s katero smo primerjali rezultate modela $M_{sp}$, smo vzeli rezultate modela $M_0$, ki o osebkih ne predpostavlja drugega, kot da so vsi enako ulovljivi. Tak model je primeren za raziskave, kjer se osebki gibljejo samo znotraj območja vzorčenja. V naravi si lahko tak primer predstavljamo na primeru ribnika, kjer ribe ne morejo zapustiti območja vzorčenja in ima vsak osebek teoretično enako verjetnost ulova. Za primerjavo, model $M_{sp}$ preko individualne spremenljivke do neke mere upošteva kršenje predpostavke neprehajanja roba območja vzorčenja.

\subsection{PRIMERJAVA AICc ZA MODELA $M_0$ IN $M_{sp}$}
Primerjava modelov $M_0$ in $M_{sp}$ kaže, da ima slednji v povprečju konsistentno nižji AICc od prvega. V primerih, ko je boljši model $M_{sp}$, je razlika v AICc med obema modeloma okoli 10. Ko je model $M_0$ boljši, je razlika največ približno 2. Razlika v AICc 2 med dvema modeloma daje podporo tezi, da sta modela tako različna, da bi lahko tistega z nižjim AICc smatrali kot boljšega \citep{boulanger_corrigendum_2001}. Modeli, ki so v AICc za 10 in več boljši od drugega (imajo nižji AICc) imajo močno podporo, da so boljši od alternativnega modela \citep{burnham_model_2002}. Iz tega lahko sklepamo, da je model $M_{sp}$ v tem pogledu v povprečju zares boljši od $M_0$. Vendar je razlika v oceni parametra ulovljivosti $p$ za praktične potrebe majhna. Končni pokazatelj, ali je ta model res pomembneje boljši, je izračun gostote osebkov v območju vzorčenja, kar je navadno cilj raziskav.

\subsection{PRIMERJAVA OCEN GOSTOT}
Izračun gostote je za klasične raziskave lova-ponovnega ulova po svoje problematičen, ker je ključen del enačbe površina območja vzorčenja. Ocena velikosti populacije se nanaša na superpopulacijo, ki lahko prehaja rob vzorčenega območja, ni pa popolnoma znano, kolikšno je to prehajanje roba. Taka ocena gostote je zato pristranska, če ne poznamo mej območja, s katerega prihaja superpopulacija. Ta problem je reševal že \citet{dice_census_1938}, tako da je območje vzorčenja razširil za polmer domačega okoliša. Na podoben način so problem poskušali reševati vsi kasnejši raziskovalci \citep{williams_analysis_2002}.

Naš popravek spada med t. i. ad hoc pristope, kot jih kritizirajo \citet{royle_spatial_2013}. Klasični pristopi po večini lovijo na mreži pasti in statistike, ki jih je moč izluščiti iz takih podatkov, so navadno na mnogo bolj grobi skali, ki jo določa razmak med pastmi. Za razliko od klasičnih pristopov, ki beležijo samo ulov oz. neulov, lahko iz podatkov, nabranih zvezno v prostoru, izluščimo tudi točne lokacije ulovov. Ob zadostnem številu točk lahko sklepamo na velikost (in obliko) povprečnega domačega okoliša. To informacijo uporabimo, da izračunamo individualno spremenljivko, ki ponazarja verjetnost, da osebek ujamemo znotraj vzorčenega območja. Osebki, ki se nikoli ne približajo robu območja vzorčenja, imajo to vrednost 1, vsi ostali pa manj, odvisno od tega, kako daleč od roba se nahaja centroid domačega okoliša in kakšna je njegova oblika.

\subsubsection[\bfseries Računanje gostote brez popravka]{Računanje gostote brez popravka}
Ko smo izračunali gostoto brez popravka velikosti območja vzorčenja, tako da smo oceno velikosti populacije delili s površino vzorčenega območja, je bila ocena vedno precenjena za vsa razmerja med velikostjo domačega okoliša in območja vzorčenja in za vse modele ($M_0$, $M_{sp}$ in TIRM). Precenjena je na račun tega, da metoda lova-ponovnega ulova ocenjuje superpopulacijo \citep{white_capture-recapture_1982}, katere osebki prehajajo območje vzorčenja.

To je tudi razlog, da se raziskovalci odločajo za popravek velikosti območja vzorčenja. Popravek je še posebej potreben v primerih, ko se s populacijo aktivno upravlja in bi jo prevelik odvzem lahko pahnil v spiralo izumiranja.

Na variabilnost ocen imata velik vpliv število odlovnih intervalov in število generiranih osebkov, ne glede na to, s katerim modelom smo ocenili velikost populacije. To verjetno ni presenetljivo, saj z več odlovnimi intervali oz. osebki v model vključujemo več podatkov, zaradi česar je ocena bolj natančna.

\subsubsection[\bfseries Računanje gostote s pomočjo popravljene velikosti območja vzorčenja]{Računanje gostote s pomočjo popravljene velikosti območja vzorčenja}
Poleg naivne gostote smo izračunali tudi gostoto, kjer smo območja vzorčenja povečali za določeno razdaljo. Za izračun te razdalje smo histogramu parnih razdalj posameznih osebkov prilegli triparametrično Weibullovo porazdelitev oz. smo predpostavili normalno porazdelitev z vrednostjo standardnega odklona iz simulacije. Na podlagi teh porazdelitev smo določili razdaljo, ki sovpada s 50. do 99. percentilom parnih razdalj. Za dano razdaljo smo povečali premer, in s tem površino, vzorčenega območja. Območje smo razširili še za dve statistiki, simuliran polmer domačega okoliša (hr) in najdaljšo parno razdaljo v simulaciji (effect).

Ocenjena gostota je za posamezen model ($M_0$, $M_{sp}$ in TIRM) konsistentna glede na razmerje med površino domačega okoliša in območja vzorčenja. To pomeni, da kažejo rezultati podobna gibanja glede na spremenljivke (število odlovnih intervalov, število generiranih osebkov, razširitev območja vzorčenja) in v odnosu do drugih modelov v vseh scenarijih. Model TIRM je najbolj precenjeval gostoto, vendar so te razlike za vse popravke v primerjavi z naivno gostoto relativno majhne.

Na sliki \ref{sli:slika6}, kjer sta prikazana oba načina računanja individualne spremenljivke, lahko vidimo, da se natančnost ocene modelov spreminja glede na število generiranih osebkov \citep{eberhardt_using_1990}. Ta učinek je še posebej viden na slikah \ref{sli:slika7} in \ref{sli:slika8}, kjer podatke prikažemo še glede na odlovne intervale. Večje število odlovnih intervalov vpliva na natančnost ocene, ker smo v model prek več odlovnih intervalov vključili več podatkov. Ta učinek je viden šele po tem, ko območje vzorčenja razširimo za katero koli mero.

Modela $M_0$ in $M_{sp}$ dajeta glede na simulirane spremenljivke (število odlovnih intervalov, število generiranih osebkov, razširitev območja vzorčenja) podobne rezultate in bi težko zagovarjali, da je v tem pogledu eden boljši od drugega. Razlike opazimo pri naivni gostoti, kjer TIRM model navadno ocenjuje nad Hugginsovima modeloma ($M_0$ in $M_{sp}$), ki sta si relativno podobna. Ko območje vzorčenja razširimo, se razlike med Hugginsovima modeloma nekoliko zabrišejo.

Bolj kot povečamo površino območja vzorčenja glede na domači okoliš osebka, manjše so razlike v oceni gostote med modeloma $M_0$ in $M_{sp}$. Naivno ocenjena gostota je najbolj pristranska in je v primerih, ko je razmerje med površino domačega okoliša in površino območja vzorčenja relativno veliko, precenjena tudi do 4-krat. To kaže, da je ocena gostote brez razmisleka o kršenju predpostavk lahko napačna, kar je pri upravljanju s prosto živečimi populacijami lahko problematično. Ker gre za precenjevanje, je treba odločevalcem pri podajanju rezultatov poudariti, da gre za optimistično oceno.

Ko območje vzorčenja razširimo za velikost simulirane velikosti domačega okoliša (en standardni odklon od centroida glede na normalno porazdelitev), se površina, s pomočjo katere računamo gostoto, poveča in posledično se ocena gostote in pristranskost zmanjšata. Najbolj točna ocena in relativno nepristranska za vse modele je okoli razširitve območja vzorčenja za 60. in 70. percentil. Če območje povečamo preveč, je ocena gostote lahko podcenjena. To velja za primere, ko za izračun individualne spremenljivke in posledično razdaljo za razširitev območja vzorčenja uporabimo znano, simulirano vrednost, ki opisuje gibanje osebkov in je t. i. ``zlati standard''. Ko območje vzorčenja razširimo za razdaljo med 60. in 70. percentilom normalne porazdelitve, zaobjamemo velik delež območja, ki ga zasedajo osebki, ki imajo centroid gibanja blizu roba območja vzorčenja. To bi približno sovpadalo s polovico premera velikosti domačega okoliša, kot je to predlagal \citet{dice_census_1938}.

Za naš popravek smo uporabili še empirično porazdelitev, ki smo jo prilegli podatkom - histogramu razdalj med pari točk. Percentil iz posamezne porazdelitve smo uporabili za izračun povečanja območja vzorčenja in izračun individualne spremenljivke.

Funkciji (empirična in normalna) se rahlo razlikujeta in razdalje pripadajočim percentilom niso identične. Empirična porazdelitev hitreje upade proti vrednosti 0 kot normalna. To je pričakovano, saj smo parametre za to porazdelitev ocenili iz vzorca lokacij osebka (za normalno porazdelitev smo uporabili simuliran standardni odklon). Podoben učinek opazimo pri metodi zankanja, ki navadno daje manjše intervale zaupanja statistik v primerjavi s parametričnimi metodami \citep{hesterberg_what_2015}. Empirična porazdelitev je tako le ocenjen približek iz podatkov. Ta ugotovitev nam služi kot varovalka, da se zavedamo, da so podatki le manjši izsek realnosti in ni nujno, da odražajo pravo stanje.

Za normalno porazdelitev tako poznamo pravo gibanje, stohastičen je samo položaj centroida domačega okoliša. Razlike, ki jih vidimo med modeloma $M_0$ in $M_{sp}$, lahko pripišemo razlikam funkcij, ki jih uporabimo za računanje individualne spremenljivke. Odstopanje centroida, izračunanega na podlagi vzorčenih točk od simuliranega gre pripisati vzorčenju majhnega števila točk.

Razlike v AICc med modeloma $M_0$ in $M_{sp}$ so primerljive s tistimi za normalno porazdelitev. Vpliv števila odlovnih intervalov ($K$) je podoben kot za normalno porazdelitev, kjer se varianca ocene gostote manjša z večanjem števila odlovnih intervalov.

Popravljanje gostote s širjenjem velikosti domačega okoliša sicer pomaga pri bolj pravilni oceni gostote, a šele ko območje vzorčenja povečamo za 80. percentil empirične porazdelitve (slika \ref{sli:slika8}). Pristranskost je, tako kot za normalno porazdelitev, relativno majhna za primere, kjer je razmerje med velikostjo domačega okoliša in območja vzorčenja relativno majhno (relativno majhen učinek roba). Pri povečanju za 90. percentil in več že prihaja do podcenjevanja ocene gostote. Najbolj točni so rezultati za simulacije z robnimi parametri - za največja in najmanjša razmerja med velikostjo domačega okoliša in območja vzorčenja. Podcenjevanje je z vidika upravljanja verjetno bolj zaželeno kot precenjevanje, ker daje zadržano oceno velikosti populacije.

Podatke smo simulirali s pomočjo normalne porazdelitve z znano varianco oz. standardnim odklonom. Ko smo računali individualno spremenljivko za posamezen osebek s pomočjo normalne porazdelitve, smo za vrednost standardnega odklona uporabili simulirano vrednost. Za popravek gostote smo območje vzorčenja razširili za razdaljo 60. percentila (približno en standardni odklon), kjer je bila gostota še najbližja simulirani. Z empirično porazdelitvijo smo to dosegli šele pri razširitvi območja vzorčenja za razdaljo 80. percentila. Iz tega bi lahko sklepali, da moramo območje povečati za določeno razdaljo, za katero, pa iz samih podatkov verjetno težko ocenimo. Morebitna rešitev bi bila z uporabo telemetrije (ali pridobimo podatek iz literature), kjer bi lahko območje vzorčenja povečali za toliko, da zajame večji delež domačih okolišev osebkov, ki so ob robu območja vzorčenja in še nezanemarljivo prispevajo k oceni številčnosti superpopulacije in gostote. Ta pristop ne bi potreboval izračuna individualne spremenljivke in so se ga poslužili že drugod \citep{ivan-et-al-2013-aux}. Nekateri sicer napovedujejo težave pri ocenjevanju parametra velikosti populacije v primeru heterogenosti ulovljivosti \citep{link_nonidentifiability_2003}, a glede na naše rezultate obstajajo primeri, ko napoved ni nujno tako slaba.

Popravek velikosti domačega okoliša bo najbolj koristil študijam, ki imajo opravka z vrstami z relativno velikim domačim okolišem (denimo medved \citep{miller_brown_1997, whittington_comparison_2015}), območje vzorčenja pa je relativno majhno. Če je območje vzorčenja znatno večje od velikosti domačega okoliša (npr. v primeru voluharjev \citealp{erlinge_density-related_1990}) in je torej razmerje blizu 0, popravek velikosti območja vzorčenja na gostoto ne bo imel bistvenega vpliva.

\newpage
\section{SKLEPI}
Ugotovili smo, da je model $M_{sp}$ z dodano individualno spremenljivko v povprečju boljši od modela, kjer spremenljivke nismo uporabili ($M_0$). Razlike v AICc so relativno velike (do približno 20) v prid $M_{sp}$. Le v nekaterih primerih je bil boljši model $M_0$, a so relativne razlike med modeloma razmeroma majhne (približno dve), do česar lahko pride že zgolj po naključju.

Ocenjeni ulovljivosti po modelu $M_{sp}$ in $M_0$ sta nižji od simulirane, kar pomeni, da zaznamo krnjenje zaradi prehajanja roba območja vzorčenja, s prvim modelom morda nekoliko bolje. Vendar pa so te razlike s praktičnega vidika zanemarljivo majhne.

Med ocenami gostote modelov $M_{sp}$, $M_0$ in TIRM nismo zaznali praktično pomembnih razlik. Z vključitvijo individualne spremenljivke v model $M_{sp}$ nismo uspeli za praktične potrebe izboljšati ``naivne'' ocene gostote. Pokazali smo, da s povečanjem območja vzorčenja na podlagi ocene velikosti domačega okoliša lahko zmanjšamo pristranskost ocene gostote.

S pomočjo simulacij smo pokazali, da je razmerje med velikostjo domačega okoliša in velikostjo območja vzorčenja pomemben dejavnik pri proučevanju učinka roba. Z manjšanjem razmerja med velikostjo domačega okoliša in območja vzorčenja ($\rightarrow$ 0) se posledice učinka roba res zmanjšujejo. To se odraža na pristranskosti ocen parametrov ulovljivosti in posledično na velikosti populacije. Za večja razmerja ($\rightarrow$ 1) lahko do neke mere učinek roba popravimo s pomočjo podatkov, ki jih dobimo o velikosti domačega okoliša osebkov. Da bomo lahko zanesljivo popravili posledice učinka roba, bo potrebno oceniti velikost oziroma parametre funkcije domačega okoliša s pomočjo več podatkov (npr. telemetrija ali iz literature).
