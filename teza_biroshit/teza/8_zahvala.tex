\section*{ZAHVALA}
\addcontentsline{toc}{section}{ZAHVALA}
\pagenumbering{gobble}

Doktorski študij je delno sofinancirala Evropska unija, in sicer iz Evropskega socialnega sklada. Sofinanciranje se izvaja v okviru Operativnega programa razvoja človeških virov za obdobje 2007-2013, 1. razvojne prioritete Spodbujanje podjetništva in prilagodljivosti; prednostne usmeritve\_1. 3: Štipendijske sheme.

Omenjeno sofinanciranje je bilo v obliki neobdavčenih 29.350,00 EUR. Odšteti gre še zneske, ki sem jih zapravil za priporočeno pošiljanje dokazil in bančnih potrdil na Univerzo.

Davnega leta 2008 me je za analizo podatkov v okviru diplomske naloge Jernej 'yerpo' Polajnar usmeril k programu \Q{R}. Program in skriptni jezik sta se mi dopadla in programiranje pozno v noč na domačiji Štelcar ni bilo nič nenavadnega. Pri analizi mi je bil v veliko pomoč Joris Meys z gentske univerze v Belgiji, ki je poskrbel, da nisem obupal že na začetku. Na srečo sem na Oddelku za biologijo našel še nekaj zanesenjakov in skupaj z Maartenom de Grootom in Majo Zagmajster smo ustanovili skupino navdušencev nad \Q{R}-jem, ki sem jo poimenoval \Q{R}koholiki. Na enem od srečanj je do mene pristopil Tomaž Skrbinšek in mi, še preden sem diplomiral, ponudil delo na projektu, ki ga je vodil Peter Trontelj. Z zamahom peresa sta mi dala edinstveno priložnost, da sem hobi zamenjal za poklic, kjer sem lahko programiral in si s tem še rezal kruh. Preko \Q{R}koholikov sem spoznal Ano Kolar, ki mi je namignila, da ``bi pa lahko vpisal doktorat iz statistike''. To je bilo verjetno prvič, da sem se sploh poigral z mislijo statistike. Leta 2011 sem vpisal doktorat. Hip hip strašen trik in projekta je končno konec. Pa hvala za vse ribe.

Veliko prijateljev mi je stalo ob strani in mi pomagalo v času doktorata. Težko bi poimensko omenil vse in jim naklonil pravično mesto med zvezdami digitalne univerzitetne knjižnice. Izpostavil pa bi svojo družino, ki mi je v dobrih in slabih časih brezkompromisno stala ob strani, za kar ji bom hvaležen do britofa. Hvala Mateji M za hitro in temeljito lektoriranje.

Brez vseh zgoraj naštetih in nenaštetih bi bil danes morebiti dosegljiv na telefonsko številko, ki jo imam v telefonu shranjeno pod ``A1 zaprti'' (01 587 22 67, Studenec 48, 1260 Ljubljana). Obljubim, da obstaja popolnoma racionalna razlaga, zakaj imam to številko shranjeno v telefonskem imeniku. Hvala.
