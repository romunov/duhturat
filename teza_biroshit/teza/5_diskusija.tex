\section{Razprava}

Pomembno orodje za raziskovanje nekaterih osnovnih parametrov populacij, kot je njena velikost, so metode ulova-ponovnega ulova. Kot velja za vse metode, imajo tudi te nekatere predpostavke, ki nas do neke mere omejujejo. V tem delu smo se osredotočili na kršenje predpostavke enake ulovljivosti. Do kršenja te predpostavke pride, ker osebki prihajajo in odhajajo iz območja vzorčenja, kar imenujemo učinek roba.

S pomočjo simulacij in klasičnih modelov za zaprte populacije (po Hugginsu) smo preverili delovanje predlaganega popravka. Vsakemu osebku smo izračunali individualno spremenljivko, ki je odvisna od lege vzorcev (centroida) v prostoru in predpostavljene funkcije gibanja. Gibanje okoli centroida smo opisali s pomočjo dveh porazdelitev. Dvorazsežno normalno porazdelitev, za katero smo uporabili simuliran standardni odklon.  Ta primer predstavlja t.i. zlati standard, saj bi morala najbolje opisati gibanje osebkov. Druga porazdelitev pa je odvisna od premikov osebkov, katerim priležemo rahlo spremenjeno kumulativno porazdelitev Weibullove funkcije. Ta način ne zahteva dodatnih informacij, ki bi jih sicer lahko pridobili z dodatnimi metodami (npr. opremljanje osebkov s telemetrijskimi ovratnicami).

Opis gibanja po (pol)normalni porazdelitvi je v simulacijskih študijah relativno pogosto \citep{bolker_ecological_2008, ivan_using_2013-1}, saj omogoča razumevanje gibanja in je hkrati (lahko) dovolj kompleksno, da opiše neke splošne lastnosti osebkov, kot se gibljejo v okolju ali naravi. Seveda pa je to poenostavitev, za katere ni nujno, da opiše gibanje osebkov dovolj dobro. Za bolj kompleksna gibanja raziskovalci uporabljajo realne podatke, ki jih naključno vzorčijo in simulirajo pseudopravilno gibanje osebkov (npr. \citet{manning_estimating_2010}). Tako generirani podatki nam res dajo bolj realen opis gibanja, vendar pa je možno, da dobimo zaradi vzorčenja nereprezentativne rezultate. Tak opis gibanja ni nujno prenosljiv med sezonami ali območji.

V tem delu smo primerjali različne scenarije, kjer smo spreminjali velikost domačega okoliša, število odlovnih presledkov, število simuliranih osebkov (gostoto) in ulovljivost. Velikost in obliko vzorčenega območja smo ohranjali konstantno, ker nas v simulacijah bolj zanima relativen odnos med velikostjo domačega okoliša in velikostjo območja vzorčenja. Rezultate simulacij smo primerjali s pomočjo treh modelov. Dva sta Hugginsova modela, $M_0$, kjer predpostavljamo enako ulovljivost vseh osebkov in $M_{sp}$, kjer predpostavljamo, da se ulovljivost spreminja z individualno spremenljivko. Tretji model je TIRM, ki predpostavlja, da imajo osebki eno od dveh ulovljivosti.

\subsection{Pristranskost ocene ulovljivosti}
Če bi območje vzorčenja zajemalo celotno območje na katerem se gibajo osebki, bi bili vsi ulovi znotraj območja vzorčenja. To v mnogih primerih ni realna možnost in se marsikater osebek giba v in iz območja vzorčenja. Pri lovu v simulaciji smo v primeru, da smo tak osebek ujeli zunaj območja vzorčenja, ta vzorec krnili. Brez krnjenja bi bila ocenjena ulovljivost nepristranska, blizu simulirane vrednosti zaznavnosti. Ker pa smo točke, ki se niso nahajale znotraj območja vzorčenja krnili, smo to ulovljivost spremenili - jo zmanjšali. To je razvidno iz rezultatov (npr. slika \ref{sli:slika5}), kjer se odmik ocenjene ulovljivosti odmakne od simulirane, predvsem na račun velikosti domačega okoliša v povezavi z velikostjo območja vzorčenja. Ko je domač okoliš relativno majhen, so te kršitve relativno majhne, a še vedno prisotne.

Ker gre za razmerje med ocenjeno in simulirano vrednostjo, so vrednosti bližje 1 bolj podobne simulirani vrednosti. Odstopanje od simulirane vrednosti je stalno glede na uporabljen model in je manjše za model $M_0$ (bližje pravi simulirani vrednosti) kot za model $M_{sp}$.\\
Prvi model je preprost, za vse osebke predpostavi enako ulovljivost in "ne ve", da so bili podatki krnjeni. Drugi model "ve", da smo podatke krnili in do neke mere za to tudi nadoknadi s pomočjo individualne spremenljivke. Ta spremenljivka naj bi opisovala pojav zaradi česar ima posamezen osebek spremenjeno ulovljivost. V našem primeru opisuje do kakšne mere osebki niso na voljo za vzorčenje. V tem oziru kaže, da individualna spremenljivka izboljša model. Ocena se izboljša predvsem za simulacije z manj osebki in manj odlovnimi presledki. Razlika med modeloma je komaj opazna, ko pa število osebkov in število odlovnih presledkov povečujemo, razlik med modeloma praktično ni več.

Odstopanje od simulirane vrednosti je različno. Odstopanje za 75 \% ni nenavadno za tiste simulacije, kjer je bila površina domačega okoliša približno tako velika kot območje vzorčenja. S pomanjševanjem površine domačega okoliša relativno na površino območje vzorčenja (razmerje se manjša) pa se ta ocena izboljšuje in v najboljšem primeru odstopa le približno 25 \%. Zaradi krnenja je manj verjetno, da bi ocenjene vrednosti (v povprečju) kadarkoli dosegle teoretično vrednost 1.

Iz razmerja ocenjene in simulirane ulovljivosti lahko opazimo, da se variabilnost zmanjša, ko smo povečali število odlovnih presledkov, nekoliko pa tudi zaradi večjega števila simuliranih osebkov. To se zgodi, ko v model vključimo več podatkov. Takrat se ocene bolj natančne in je posledično oceni parametrov manj variabilna. Ta pojav nam je lahko v pomoč pri načrtovanju poskusa, kjer se odločamo, ali napor vložiti v večanje števila ulovljenih osebkov, v večanje števila odlovnih presledkov ali obojega.

Kot osnovo, s katero smo primerjali rezultate modela Msp, smo vzeli rezultate modela $M_0$, ki o osebkih ne predpostavlja drugega, kot da so vsi enako ulovljivi. Tak model je primeren za raziskave, kjer se osebki gibljejo samo znotraj območja vzorčenja. V naravi si lahko tak primer predstavljamo na primeru ribnika, kjer ribe ne morejo zapustiti območja vzorčenja in ima vsak osebek teoretično enako verjetnost ulova. V primerjavi, model $M_{sp}$, ki preko individualne spremenljivke do neke mere upošteva kršenje predpostavke neprehajanja roba območja vzorčenja.

\subsection{Primerjava AICc za modela $M_0$ in $M_{sp}$}
Primerjava modelov $M_0$ in $M_{sp}$ kaže, da ima slednji v povprečju konsistentno nižji AICc od prvega. V primerih, ko je boljši model $M_{sp}$ je razlika v AICc med obema modeloma okoli 10. Ko je model $M_0$ boljši, je razlika največ približno 2. Razlika v AICc 2 med dvema modeloma daje podporo tezi, da sta modela tako različna, da bi lahko tistega z nižjim AICc smatrali kot boljšega \citep{boulanger_corrigendum:_2001}. Modeli, ki so v AICc za 10 in več boljši od drugega (imajo nižji AICc) imajo močno podporo, da so boljši od alternativnega modela \citep[stran 71]{burnham_model_2002}. Iz tega lahko sklepamo, da je model $M_{sp}$ v tem pogledu v povprečju zares boljši od $M_0$.

Končni pokazatelj, če je ta model res pomembneje boljši pa je izračun gostote osebkov v območju vzorčenja, kar je navadno cilj raziskav.

\subsection{Primerjava ocen gostot}
Izračun gostote je za klasične raziskave ulova-ponovnega ulova po svoje problematičen, ker je del enačbe površina območja vzorčenja. Ocena velikosti populacije se nanaša na superpopulacijo, ki lahko prehaja rob vzorčenega območja, ni pa popolnoma znano kolikšno je to prehajanje roba. Taka ocena gostote je zato pristranska, če ne poznamo meja območja s katerega prihaja superpopulacija. Ta problem je reševal že \citep{dice_census_1938} tako, da je območje vzorčenja razširil za polmer domačega okoliša. Na podoben način so poskušali reševati vsi raziskovalci po njem \citep{williams_analysis_2002}.

Naš popravek spada med ti.i. ad hoc pristope, kot jih kritizira \citep{royle_spatial_2013}). Klasični pristopi po večini lovijo na mreži pasti in statisitke, ki jih je moč izluščiti iz takih podatkov so navadno na mnogo bolj grobi skali, ki jo določa razmak med pastmi. Za razliko od klasičnih pristopov, ki beležijo samo ulov oz. neulov, lahko iz podatkov nabranih zvezno v prostoru izluščimo tudi točne lokacije ulovov. Ob zadostnem številu točk lahko sklepamo na velikost (in obliko) povprečnega domačega okoliša. To informacijo uporabimo, da izračunamo individualno spremenljivko, ki ponazarja verjetnost, da osebek ujamemo znotraj vzorčenega območja. Osebki, ki se nikoli ne približajo robu območja vzorčenja, imajo to vrednost 1, vsi ostali pa manj, odvisno od tega kako daleč roba se nahaja centroid domačega okoliša in kakšna je njegova oblika.
Računanje gostote brez popravka
Ko smo izračunali gostoto brez popravka velikosti območja vzorčenja, tako da smo oceno velikosti populacije delili s površino vzorčenega območja, je bila ocena vedno precenjena za vsa razmerja med velikostjo domačega okoliša in območja vzorčenja in vse modele ($M_0$, $M_{sp}$ in TIRM). Precenjena je na račun tega, ker metoda ulova-ponovnega ulova ocenjuje superpopulacijo \citep{white_capture-recapture_1982}, katere osebki prehajajo območje vzorčenja.

To je tudi razlog, da se raziskovalci odločajo za popravek velikosti območja vzorčenja. Popravek je še posebej potreben v primerih, ko se s populacijo aktivno upravlja in bi jo prevelik odvzem lahko pahnil v spiralo izumiranja.

Na variabilnost ocen imata velik vpliv število odlovnih presledkov in število generiranih osebkov, ne glede na to s katerim modelom smo ocenili velikost populacije. To verjetno ni presenetljivo, saj z več odlovnimi intervali oz. osebki v model vključujemo več podatkov zaradi česar je ocena bolj natančna.

\subsubsection{Računanje gostote s pomočjo popravljene velikosti območja vzorčenja}
Poleg naivne gostote smo izračunali tudi gostoto, kjer smo območja vzorčenja povečali za določeno razdaljo. Za izračun te razdalje smo histogramu parnih razdalj posameznih osebkov prilegli triparametrično Weibullovo porazdelitev oz. smo predpostavili normalno porazdelitev z vrednostjo standardnega odklona iz simulacije. Na podlagi teh porazdelitev smo določili razdaljo, ki sovpada s 50. do 99. percentilom prehojenih razdalj. Za dano razdaljo smo povečali premer, in s tem površino, vzorčenega območja. Območje smo razširili še za dve statistiki, simuliran polmer domačega okoliša (hr) in za najdaljšo prehojeno razdaljo v simulaciji (effect).

Ocene gostote za posamezen model ($M_0$, $M_{sp}$ in TIRM) konsistentna glede na razmerje med površino domačega okoliša in območja vzorčenja. To pomeni, da kažejo podobna gibanja glede na spremenljivke (število odlovnih intervalov, število generiranih osebkov, razširitev območja vzorčenja) in v odnosu do drugih modelov v vseh scenarijih. Model TIRM je najbolj precenjeval gostoto, vendar so te razlike za vse popravke glede na naivno gostoto relativno majhne.

Na sliki \ref{sli:slika6}, kjer sta prikazana oba načina računanja individualne spremenljivke, lahko vidimo, da se natančnost ocene modelov spreminja glede na število generiranih osebkov \citep{eberhardt_using_1990}. Ta učinek je še posebej viden na slikah \ref{sli:slika7} in \ref{sli:slika8}, kjer podatke prikažemo še glede na odlovne presledke. Večje število odlovnih presledkov vpliva na natančnost ocene, ker smo v model preko odlovnih presledkov vključili več podatkov. Ta učinek je viden šele po tem, ko območje vzorčenja razširimo za katero koli mero.

Modela $M_0$ in $M_{sp}$ dajeta glede na simulirane spremenljivke (število odlovnih presledkov, število generiranih osebkov, razširitev območja vzorčenja) podobne rezultate in bi težko zagovarjali, da je v tem pogledu eden boljši od drugega. Razlike opazimo pri naivni gostoti, kjer TIRM model navadno ocenjuje nad Hugginsovima modeloma, le-ta ($M_0$ in $M_{sp}$) pa sta si relativno podobna. Ko območje vzorčenja razširimo, se razlike med Hugginsovima modeloma nekoliko zabrišejo.

Bolj kot povečamo površino območje vzorčenja, manjše so razlike v oceni gostote med modeloma $M_0$ in $M_{sp}$. Naivno ocenjena gostota je najbolj pristranska in je v primerih, ko je razmerje med površino domačega okoliša in površino območja vzorčenja relativno veliko, precenjena tudi do 4-krat. To kaže na to, da je ocena gostote brez razmisleka o kršenju predpostavk lahko napačna, kar je pri upravljanju s prosto živečimi populacijami lahko problematično. Ker gre za precenjevanje, je potrebno odločevalcem pri podajanju rezultatov poudariti, da je to optimistična ocena.

Ko območje vzorčenja razširimo za velikost simulirane velikosti domačega okoliša (en standardni odklon od centroida glede na normalno porazdelitev) se površina s pomočjo katere računamo gostoto poveča in posledično se ocena gostote in pristranskost zmanjša. Najbolj točna ocena in relativno nepristranska za vse modele je okoli razširitve območja vzorčenja za 60. in 70. percentil. Če območje povečamo preveč, je ocena gostote lahko podcenjena. To velja za primere, ko za izračun individualne spremenljivke in posledično razdaljo za razširitev območja vzorčenja uporabimo znano, simulirano vrednost, ki opisuje gibanje osebkov in je t.i. "zlati standard". Ko območje vzorčenja razširimo za razdaljo med 60. in 70. percentilom normalne porazdelitve, zaobjamemo velik delež območja, ki ga zasedajo osebki, ki imajo centroid gibanja blizu roba območja vzorčenja. To bi približno sovpadalo s polovico premera velikosti domačega okoliša, kot je to predlagal \citep{dice_census_1938}.

Za naš popravek smo uporabili še empirično porazdelitev, ki smo jo prilegli podatkom - histogramu razdalj med pari točk. Percentil iz posamezne porazdelitve smo uporabili za izračun povečanja območja vzorčenja in izračun individualne spremenljivke.

Funkciji (empirična in normalna) se rahlo razlikujeta in njuni percentili niso identični. Empirična porazdelitev hitreje upade proti vrednosti 0 kot normalna. To je pričakovano, saj smo parametre za to porazdelitev ocenili iz vzorca lokacij osebka (za normalno porazdelitev smo uporabili pravo, simulirano, standardi odklon). Podoben učinek opazimo pri metodi zankanja, ki navadno dajejo manjše intervale zaupanja statistik v primerjavi s parametričnimi metodami \citep{hesterberg_what_2015}. Empirična porazdelitev je tako le ocenjen približek iz podatkov. Ta ugotovitev nam služi kot varovalka, da se zavedamo, da so podatki le manjši izsek realnosti in za katere ni nujno, da odražajo pravo stanje.

Za normalno porazdelitev tako "vemo" pravo gibanje, stohastičen je samo centroid domačega okoliša. Razlike, ki jih vidimo med modeloma $M_0$ in $M_{sp}$ lahko pripišemo razlikam funkcij, ki jo uporabimo za računanje individualne spremenljivke in na račun odstopanj centroidov od simulirane vrednosti.

Razlike v AICc med modeloma $M_0$ in $M_{sp}$ so primerljive s tistimi za normalno porazdelitev. Vpliv števila odlovnih presledkov ($K$) je podoben kot za normalno porazdelitev, kjer se varianca ocene gostote manjša z večanjem števila odlovnih presledkov.

Popravljanje gostote s širjenjem velikosti domačega okoliša sicer pomaga, a nikakor v taki meri kot smo to opazili za normalno porazdelitev (slika \ref{sli:slika8}). Pristranskost je, tako kot za normalno porazdelitev, relativno majhna za primere, kjer je razmerje med velikostjo domačega okoliša in območja vzorčenja relativno majhno (relativno majhen učinek roba). Nepristranskost gostote se občutno zmanjša šele za primere, ko območje vzorčenja razširimo za približno 80.-90. percentil. Pri 90. percentilu in več že prihaja do podcenjevanja ocene gostote. Najbolj točni so rezultati za ekstreme simulacije - za največja in najmanjša razmerja med velikostjo domačega okoliša in območja vzorčenja. Podcenjevanje je z vidika upravljanja verjetno bolj zaželeno kot precenjevanje ker daje to zadržano oceno velikosti populacije.

\section{Zaključki}
Ugotovili smo, da je model $M_{sp}$ (model z dodano individualno spremenljivko) v povprečju boljši od modela, kjer te spremenljivke nismo uporabili ($M_0$). Razlike v AICc so lahko relativno velike (do $\sim 20$), v prid $M_{sp}$. V samo nekaterih primerih je bil boljši model $M_0$, a so relativne razlike med modeloma relativno majhne (do $\sim 2$), do česar lahko pride že zgolj po naključju.

Ucenjena ulovljivost po modelu $M_{sp}$ je manj pristranska - nižja od simulirane, kar pomeni, da zaznamo krnenje zaradi prehajanje roba območja vzorčenja. Vendar pa so te razlike relativno majhne.

Med ocenama gostote modelov $M_{sp}$, $M_0$ in TIRM nismo zaznali praktično pomembnih razlik. Z vključitvijo individualne spremenljivke v model Msp nismo uspeli za praktične potrebe izboljšati ocene gostote.

S pomočjo simulacij smo pokazali, da je razmerje med velikostjo domačega okoliša in velikostjo območja vzorčenja pomemben dejavnik pri učinku roba. Z manjšanjem razmerja med velikostjo domačega okoliša in območja vzorčenja ($\rightarrow$ 0) se posledice učinka roba res zmanjšujejo. To se odraža na pristranskosti ocen parametrov ulovljivosti in velikosti populacije. Za večja razmerja ($\rightarrow$ 1) lahko do neke mere učinek roba popravimo s pomočjo podatkov, ki jih dobimo o velikosti domačega okoliša osebkov.
